% https://cienciapratica.wordpress.com/2015/01/10/escrevendo-o-resumo-ou-%E2%80%9Cabstract%E2%80%9D-para-um-artigo/
% resumo em português

% Conforme a ABNT NBR 6022:2003, o resumo é elemento obrigatório, constituído de
% uma sequência de frases concisas e objetivas e não de uma simples enumeração
% de tópicos, não ultrapassando 250 palavras, seguido, logo abaixo, das palavras
% representativas do conteúdo do trabalho, isto é, palavras-chave e/ou
% descritores, conforme a NBR 6028. (\ldots) As palavras-chave devem figurar logo
% tabaixo do resumo, antecedidas da expressão Palavras-chave:, separadas entre si por
% ponto e finalizadas também por ponto.

\renewcommand{\resumoname}{RESUMO}
\begin{resumoumacoluna}
	
	\lipsum[1]
	
	\noindent
	\textbf{Palavras-chave}: php. java. javascript. programação orientada a objetos
	
\end{resumoumacoluna}

% resumo em inglês
\renewcommand{\resumoname}{ABSTRACT}
\begin{resumoumacoluna}
	\begin{otherlanguage*}{english}
		
		\lipsum[2]
		
		\noindent
		\textbf{Keywords}: php. java. javascript. object oriented programming
	\end{otherlanguage*}  
\end{resumoumacoluna}